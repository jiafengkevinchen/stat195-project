\documentclass[11pt]{article}
\usepackage{macros}
\usepackage{bbm}
\usepackage{longtable}
\usepackage{titlesec}


\titleformat{\section}
  {\normalfont\sffamily\Large\bfseries}
  {\thesection}{1em}{}
\titleformat{\subsection}
  {\normalfont\sffamily\large\bfseries}
  {\thesubsection}{1em}{}
  
\titleformat{\subsubsection}
  {\normalfont\sffamily\normalsize\bfseries}
  {\thesubsubsection}{1em}{}
  
\hypersetup{urlcolor=Purple}

% \singlespacing
\newgeometry{left=1in, right=1in, bottom=1.3in}
\title{\sffamily\bfseries{Prediction of the 2018 Midterm Elections}}
\author{Jiafeng Chen\thanks{Harvard College, \url{jiafengchen@college.harvard.edu}} \and Joon Hyuk Yang\thanks{Harvard College, \url{joonhyukyang@college.harvard.edu}}}
\begin{document}
\maketitle
\section{Introduction}

\section{Data}
\subsection{Informative Prior Features}
We first consider a range of potentially relevant features to uncover an overall trend in Midterm Election results since 2010. We collect, process, then use a second-order polynomial basis transform on the features to discover nonlinear correlations in constructing our prior. The following is a complete list of the features were used: Republican or Democratic incumbency, gender, voting pattern in the previous presidential election, percentage of minorities, presidential approval rating, candidates’ party alignment with the incumbent president, log of median income, percentage of district population with a Bachelor’s degree or above, one-hot encoding of each state, and the ratio of search frequency via Google Trends.

One design choice we made in feature selection is to ignore Third-party candidates. Since 2010, there was a single Independent candidate (Jo Ann Emerson) who won a seat at the House, and given the that there was no evidence to support a sudden surge in Third-party candidates in the 2018 election, we focus on the major parties.   

\subsection{Google Trends}
Google Trends, unlike Google Keyword Planner, shows the relative popularity of search queries from a scale of 0 to 100. Relative popularity refers to the ratio of a query's search volumne to the sum of the search volumes of all possible queries. This allows for an easier and more intuitive comparison between terms via search across arbitrary periods of time. Moreover, Google Trends allows for a state-level granularity in search query trends. We take advantage of this to compare the candidates of each district by taking the ratio of their 

\subsection{Polling Data for Posterior Updates}
\section{Statistical model and prediction function}
\subsection{Overview}
Let \[Y_i = \frac{\text{Republican}\% }{\text{Republican}\% + \text{Democrat}\%}\] be the proportion of Republican vote share between the major parties in district $i$, where $i$ is indexed by the triple $(\text{state}, \text{district number}, \text{year of the race})$.
%\footnote{Assuming districts have no time dimension: i.e. Alabama-01 is
%represented by different $i,j$'s across two different years. We also assume for
%simplicity that third parties never win elections, which seems accurate in the
%case for 2018.}
Let $x_i\in \R^p$ be a list of features (after suitable basis transformations)
for the district. 

A machine learning method can generate a prediction function $\hat f_{\cal T}$,
where $\mathcal T$ is the collection of $Y_i, x_i$ over the training set.
However, the
distribution of the test set, $Y_i \mid x_i$ for those districts $i$ in 2018 may
be
substantially different from that of $Y_j \mid x_j$ for $j$ in the training
set. Thus $\hat f_{\mathcal T}$ may incur large generalization errors, since we
are attempting to generalize the prediction function to a potentially 
\emph{different}
distribution
of the data.\footnote{Note that this problem is mitigated by increased data
availability---for instance, if we had obtained historical polling data,
which is difficult to obtain---but we incur a different problem ($p > N$) of
high-dimensional
inference in that case.} Fortunately, we have an additional piece of
information, namely polling data, that we may scrape from polling aggregators
like
FiveThirtyEight.\footnote{\url{https://projects.fivethirtyeight.com/polls/}} The
lack of historical polling data means that we cannot simply include them in the
features $x_i$. Therefore, to incorporate polling data, we perform a two-step
Bayesian procedure. In the first step, we estimate a prior $Y \sim p_{\theta_0,
X}$, where parameters $\theta_0$ is estimated from the training data. In the
second step, we assume that polling data, $Z$, comes from some conditional
distribution $Z \mid Y \sim p_{\eta}(z\mid y)$ for some known or
estimated parameters
$\eta$. We form our final prediction by computing the posterior distribution $Y
\mid Z$. We now discuss how we parameterize $p_{\theta, X}$ and $p_\eta$.

\subsection{Parameterizations}

For the prior, $p_{\theta, X}(y)$, we assume a linear probability model,
\begin{equation}
    Y \sim \Norm(\underbrace{X\beta_0}_{\mu_0}, \Sigma_0)
    \label{eq:first_stage}
\end{equation}
\Cref{eq:first_stage} is misspecified since $Y_i \in [0,1]$ but the Normal
distribution is supported on $\R$. A properly specified generalized linear model
for $Y_i$ is a Beta linear model \citep[See][for an overview]{grun2011extended}.
We have a few reasons for preferring \Cref{eq:first_stage} instead. First, we
note that for large values of $a,b$ in $\Beta(a,b)$, the distribution
$\Beta(a,b)$ is approximately Normal. Second, we note that \[Y_i = M^{-1}
\sum_{j=1}^M \frac{\mathbbm{1} (\text{vote}_j \in \{D,R\})}{\sum_{j'=1}^M \mathbbm{1}
(\text{vote}_{j'} \in \{D,R\})} \mathbbm{1}(\text{vote}_j = R),\] for a district with
$M$ voters, which should have an approximately Normal distribution, by the
Central Limit Theorem. Third, computational methods for the Beta regression
model---especially those that deal effectively with high-dimensional
covariates---is much less readily available than those for the Normal linear
model. We note additionally in \TODO {Figure X} that the fitted residuals are
approximately Normally distributed, and we note that Normal distribution eases
computation for Bayesian updating in our next step. For \Cref{eq:first_stage},
we need to estimate $\theta_0 = (\beta_0, \Sigma_0)$. 

For $p_{\eta}$, we assume that conditional on $Y$, the polls $Z_j$ are
independently distributed. We can treat an observation of a poll $z_j$ as
generated by
asking $n_j$
individuals a question (``Are you voting for the Republican candidate''), with
$z_j \in [0,1]$ respondents responding
affirmatively.\footnote{Again, this model assumes that the only parties running
are Republicans and Democrats, and that there are no undecided individuals.
When working with data, we normalize the Republican percentage by the sum of
the Republican and Democratic percentages.}
Assume that 
\begin{equation}
    Z_j \mid Y \sim \Norm(a_j^T Y, \sigma_{Z_j}^2), \text{ for $a_j, \sigma_
    {Z_j}^2$ that do not depend on $Y$}.
    \label{eq:polls}
\end{equation}
We use the Normal distribution in \Cref{eq:polls} because (1) The Normal
distribution works well computationally with our Normal prior\footnote{This is
assuming that the Normal likelihood has variance unaffected by the
prior---which requires a further justification that we discuss in 
\Cref{sec:estimate}.} and (2) polling proportions are empirical averages, which
tend to be Normally distributed by the Central Limit Theorem.



We use two types of polls in our implementation. First is a national
\emph{generic ballot} poll, which simply asks the respondent which party she
would vote for. For such $Z_j$, we assume $a_j = n^{-1} \bm 1$, treating the
mean of such a poll as aggregated over all contested districts. Second is a
district-wide poll, and we use $a_j = e_i$ for a poll in district $i$, where
$e_i$ is the $i$-th standard basis vector. For \Cref{eq:polls}, we need to
estimate $\sigma_{Z_j}^2$ for both types of polls.

We now discuss our approaches to estimation and justify discretionary choices
made therein. 

\subsection{Estimation}
\label{sec:estimate}
\subsubsection{Estimation for $\theta_0$}
We first discuss estimation for $\theta_0 = (\beta_0, \Sigma_0)$.
Let \begin{equation}
\hat \beta_0 = \argmin_\beta \, \sum_{i=1}^{|\mathcal T|} w_i (y_i - x_i^T
\beta)^2
+ \lambda
\pr{\alpha \norm{\beta}_1 + (1-\alpha) \norm{\beta}_2^2}
\tag{Estimation for $\beta_0$}
\label{eq:beta}
\end{equation}
be fitted with an elastic net regularizer over the training data with weights
$w_i$,
where
$(\alpha, \lambda)$ is chosen via 10-fold cross-validation. The weights we chose is uniform $w_i = 1$; however, in the spirit of efficient
estimation as in weighted least
squares, it may be desirable to choose the weights to the proportional to total
votes cast, since we should expect that variance of the measurement $Y_i$ is
inversely proportional to the number of votes cast, as $Y_i$ is an empirical
average. In numerical experimentation, we found that the weights do not affect
results much, and we keep $w_i = 1$ for simplicity.

It is crucial, for the purpose of uncertainty quantification, that we perform good variance estimation. Let $S_i = \{j: \mathsf{state}(j) = 
\mathsf{state}(i)\}$ be those districts in the same state as $i$ (across time),
\[
\hat \Sigma_{ii} = \frac{1}{|S_i|} \sum_{j\in S_i}
\pr{y_j
- x_j^T \hat \beta_0}^2 \tag{Estimation for variance}
\label{eq:var}
\]
be an estimate for variance. Here we estimate variance by pooling over the state
of a district, as states boundaries, unlike district boundaries, do not change
over time. Alternatively, we could estimate variance by letting $S_i$ be those
districts that share the same state and district identifier (which may not be
the same geographical area due to redistricting), or with the $k$ nearest geographical neighbors of district $i$ (in the spirit of $k$NN). 

Optionally, we also estimate off-diagonal elements of $\Sigma$. If district boundaries do not change over time, then we can easily estimate off-diagonal entries of $\Sigma$ as well, since we can use the variation over time to calculate the empirical analogue of covariance: \[
\hat \Sigma_{ij} = \frac{1}{T}\sum_{t=1}^T \pr{y_{it} -x_{it}^T \hat \beta_0}\pr{y_{jt} -x_{jt}^T \hat \beta_0} \tag{Estimation for covariance}
\label{eq:covar}
\]
This approach is no longer valid when district boundaries change over time. Nonetheless, we can still calculate $\hat \Sigma_{ij}$, ignoring district boundary changes.  Note that \eqref{eq:var} is inconsistent with \eqref{eq:covar}, and filling in estimates from \eqref{eq:covar} directly into $\Sigma$ would generate non-positive-definite matrices. As a result, we estimate the \emph{correlation} $\hat \rho_{ij}$ similar to \eqref{eq:covar},\footnote{We fill undefined values of $\hat \rho_{ij}$ with zero.} and use the estimate 
 \[
\hat \Sigma_{ij} = \hat \rho_{ij} \hat \Sigma_{ii} \hat \Sigma_{jj}
 \tag{Estimation for covariance with correlation}
\label{eq:covar_corr}
\]
which ensures that the resulting $\Sigma_0$ is positive definite. 

Here we discuss a few caveats. In principle, the variance we estimate should also take into account the uncertainty in $\hat \beta_0$. However, there is little consensus in uncertainty quantification for methods like elastic net and LASSO \citep{kyung2010penalized}, which remains an active research area. The fitting (along with cross-validation) is too expensive to bootstrap either, and we ignore the uncertainty in $\hat \beta_0$ for simplicity. Moreover, \eqref{eq:var} is likely an underestimate of variance---in linear regression, for example, we inflate the naive variance estimate by $n/(n-p)$, where $p$ is the degrees of freedom in a linear regression. The analogous inflation in our setting is with the degree of freedom in the elastic net \citep{zou2007degrees}---the estimate of the degree of freedom is about 100 for a dataset with $N=1500$; thus the inflation factor is about $\frac{1500}{1500-100} \approx 1.07$ for variance, which is sufficiently small to ignore. An alternative would be to estimate variance on a hold-out set, which is unbiased if the data is i.i.d.; we do not hold out for the sake of increasing training data.  

As a notational matter, we now suppress the hat notation on $\beta_0, \Sigma_0.$ We use $\Sigma_0^d$ to distinguish the diagonal of $\Sigma_0$ from the full-estimate with \eqref{eq:covar_corr}. 
\subsubsection{Estimation for polling variance}

We now turn to estimation of variance in the likelihood of polls, $\sigma_{Z_j}^2$ in \Cref{eq:polls}. The main problem in variance estimation is to account for the large between-poll variance unpredicted by a simple model. Suppose poll outcomes are indepedently $n_j^{-1} \Bin(n_j, Y_i)$ conditional on $Y_i$, then the (conditional) standard deviation is as low as $1.58$ percentage points for a poll with $1000$ respondents. However, polls of this size display much more variation than predicted by a simple independent Binomial model. Therefore, we assume that there is some structural between-poll variance and attempt to estimate it from data.  

For a generic ballot poll $Z_j$, we leverage the \href{https://projects.fivethirtyeight.com/congress-generic-ballot-polls/}{FiveThirtyEight polling average estimates} since FiveThirtyEight provides a 90\% confidence interval. We use the (most recent) variance that implies the range of $90\%$ confidence computed by FiveThirtyEight. The FiveThirtyEight estimates are via kernel-weighted local polynomial smoothing,\footnote{The link on footnote 2 in \href{https://fivethirtyeight.com/features/heres-a-new-less-volatile-version-of-our-generic-ballot-tracker/}{FiveThirtyEight's methodology} links to the Stata manual for kernel-weighted local polynomial smoothing.} whose confidence comes from (we conjecture) bootstrapping. We directly use FiveThirtyEight's estimates for convenience. 

For a district-wide polls $Z_j^i, j = 1,\ldots,J_i$, the variance comes from two components. The first component comes from the sampling error of the poll itself, which is the Binomial variance $\frac{Y_i(1-Y_i)}{n_j} \le \frac{1}{4n_j}$. The second component comes from a between-poll variance, which we estimate by taking the empirical variance of poll results in a district.\footnote{Filling missing values with state-averages.} We use \[
\sigma^2_{Z_j} = \frac{1}{4n_j} + \sigma^2_{i,p} \tag{Estimation for district poll variance}
\]
For computational simplicity, we use the upper bound $\frac{1}{4n_j}$ as a proxy for the first part of variance, since it does not depend on $Y_i$ in order to take advantage of Bayesian conjugacy; the approximation is good for districts with $Y_i$ close to $1/2$, which is precisely those races that we especially care about. Full, proper posterior condition can still be achieved if we use the more precise $Y_i(1-Y_i)/n_j$ via an MCMC sampler. 

% Let $\hat \sigma_{0i}^2 = \sum (y_j - x_j^T \hat\beta_0)^2$ be fitted as the variance of the residuals, where the sum could be over districts in the same state, over all districts, or some kernel-weighted estimator. For simplicity, we stratify variance estimate by state.\footnote{One could also estimate variance over a holdout set, which might improve bias.} Let $\hat \mu_{i0} = x_i^T \hat \beta_0$.


% For a district that corresponds to an upcoming election, we form a prior 
% $Y_i\sim \Norm(\hat \mu_{i0}, \hat \sigma_0^2),$ or in vector form \[
% Y \sim \Norm(\hat \mu_0, \hat \Sigma_0)
% \]
% Note that in such a formulation, we ignore the sampling variance of $\hat
% \mu_{i0}$ and $\hat \sigma_0^2$,\footnote{The elastic net regularizer in the
% fitting method for $\beta$ makes the sampling variance of $\hat\beta_0$
% difficult to compute.} instead forming a plug-in estimate, appealing to the law
% of large numbers.\footnote{From this point on, we drop the hat on $\mu_{i0},
% \sigma_0$.} To obtain a more accurate and timely prediction for the district, we
% update the prior in two steps.

% First, to take into account the ``blue wave,'' we update our prior via the generic congressional ballot.\footnote{\url{https://projects.fivethirtyeight.com/congress-generic-ballot-polls/}} Formally, we model generic congressional poll as $Z_{G} \mid Y_i \sim \Norm(Y_i, \sigma_G^2),$ where $\sigma_G^2$ is estimated from the 90\% confidence interval provided by FiveThirtyEight. We have the following data generating process: \begin{align*}
% Y &\sim \Norm(\hat \mu_0, \hat \Sigma_0)\\
% Z_G \mid Y &\sim \Norm\pr{n^{-1}1^T Y, \sigma_G^2}
% \end{align*}

% We update our prior to form an intermediate posterior: \[
% Y \mid Z_G \sim \Norm(\mu_1, \Sigma_1).
% \]
% From now on, we drop the conditioning on $Z_G$.

% Second, for some districts, we observe a number of district-specific polls $Z_{i1},\ldots,Z_{iJ_i}$.\footnote{In practice, we take the $J$ most recent polls (if available) in district $i$, where $J = 10$.} The variance across polls is much higher than implied by a simple Beta-Binomial model, where one assumes that each poll is an independent $\Bin(n_j,p)$ where $p$ is sampled from a Beta (again, approximately Normal) prior. As such, we hesitate from using a simple Beta-Binomial updating procedure and opt for the following model: We assume that poll $j$ has an independent bias $\epsilon_j \sim \Norm(0, \sigma^2_p)$, where $\sigma_p$ is estimated as the empirical variance of poll outcomes in a district: \begin{align*}
% Y &\sim \Norm(\mu_1, \Sigma_1) \quad \epsilon_j \sim \Norm(0, \sigma_p^2) \\
% Z_{ij}\mid Y_i, \epsilon_j &\sim n^{-1} \cdot \Bin(n, e_i^T Y + \epsilon_j),
% \end{align*}
% where $e_i$ is the $i$th standard basis vector. It's somewhat difficult to compute the posterior $Y \mid Z_{i1}, \ldots, Z_{iJ}$. Instead we may assume the misspecified model, justified as $n^{-1}\Bin(n, p)$ is approximately Normal when $n$ is large. \begin{align*}
% Y &\sim \Norm(\mu_{1}, \Sigma_{1}) \\
% Z_{ij} \mid Y &\sim \Norm\pr{e_i^TY, \frac{1}{4n_j} + \sigma_p^2},
% \end{align*}
% so as to (a) take advantage of Normal-Normal conjugacy and (b) ignore the dependence of $\var(Z_{ij})$ on $Y_i$. We now have the posterior by, say, sequentially Bayesian updating: \[
% Y \mid (Z_{i1},\ldots,Z_{iJ_i})_{i=1}^n \sim \Norm\pr{
%     \mu_{2}, \Sigma_{2}
% }.
% \]
% We predict $\hat Y_i = \mu_{2i}$, and, naturally \[
% \hat{\text{Winner}}_i = \begin{cases}
%     \text{Republican} & \mu_{2i} > .5 \\
%     \text{Democrat} & \mu_{2i} < .5
% \end{cases}
% \]

\section{Results}

\section{Considerations}


\bibliographystyle{jpe}
\bibliography{main.bib}
\end{document}
